\pdfoutput=1

\documentclass{l4proj}

%
% put any packages here
%

\begin{document}
\title{Are matrix-based or node-linked graphs more readable when representing causal relationships for social and health data?}
\author{Kristina Lazarova}
\date{March, 2017}
\maketitle

\begin{abstract}
We show how to produce a level 4 project report using latex and pdflatex using the 
style file l4proj.cls
\end{abstract}

\educationalconsent
%
%NOTE: if you include the educationalconsent (above) and your project is graded an A then
%      it may be entered in the CS Hall of Fame
%
\tableofcontents
%==============================================================================
\chapter{Literature Review}
\pagenumbering{arabic}

This is the first chapter where I will introduce data visualisation and and explain how I came up with the idea of this research

\chapter{Introduction}
Introduction to the specific area of my research

\chapter{Implementation details}
\section{Software tools and technologies}
A web application framework 
\section{Challenges}
\begin{verbatim}
		Spring idea failed
		changed to Node.js
		Angular compatibility with Node.js
\end{verbatim}

In the beginning of this project the Java framework Spring was going to be used in the implementation as it is among the most widely used frameworks in industry \cite{shiLuiLi}. This decision was supported by extensive previous experience with Java from developer's point of view and the applicability of the skills to be acquired. However, one of the reasons why Spring is used in industry is because of the large and complex systems that exist there. The Spring framework works on a very high level of abstraction where you can easily write configuration files to add dependencies from different project. Therefore, it is considered rather unfriendly for small independent projects and developers with limited Spring experience. The reasoning behind this conclusion was provoked after a couple of unsuccessful attempts to set relative paths to different CSS and JavaScript files. The issue was found to be in the web application configuration file. This is how the very simple task of reading a css file turned to be a long tedious debugging process after which the realisation that Spring is unnecessary abstract and complex for this project occurred.

A new research for web-application frameworks followed. Node.js backend was chosen because of its event-driven, non-blocking I/O model which creates an efficient and lightweight server-side of the application. Another challenge appeared when trying to incorporate AngularJS with Node.js. Usually in AngularJS one uses curly braces to reference data structure from the AngularJS controller. However, Node.js also uses curly brackets to reference information from the backend in the frontend. After a long research  it was found that Node.js overrides the use of curly braces and the application is not displaying Angular data as it expects it come from the backend. Unfortunately, an appropriate error message does exist and it all had to be discovered during the development process. Instead of using curly brackets one can also use "ng-bind" and achieve the same result. This approached solved the issue until "ng-bind" information was need in "ng-src" to display the appropriate graph image. It is not possible to use "ng-bind" inside "ng-src" so the present solution at the time was no longer solving the problem. Therefore, the Angular configurations had be altered to use a different symbol. Implementing this solved the problem entirely. 


\section{Software reliability testing}

\chapter{Evaluation}
\section{Design}
\section{Participants}
\section{Procedure}
\section{Results}
\section{Discussion}

\chapter{}




\section{First Section in Chapter}
The quick brown fox jumped over the lazy dog.
The quick brown fox jumped over the lazy dog.
The quick brown fox jumped over the lazy dog.
The quick brown fox jumped over the lazy dog.
The quick brown fox jumped over the lazy dog \cite{DIMACS}.
The quick brown fox jumped over the lazy dog.
The quick brown fox jumped over the lazy dog.
The quick brown fox jumped over the lazy dog.
The quick brown fox jumped over the lazy dog.

\subsection{A subsection}
The quick brown fox jumped over the lazy dog.
The quick brown fox jumped over the lazy dog.
The quick brown fox jumped over the lazy dog.
The quick brown fox jumped over the lazy dog.

The quick brown fox jumped over the lazy dog.
The quick brown fox jumped over the lazy dog.
The quick brown fox jumped over the lazy dog.
The quick brown fox \cite{fahle} jumped over the lazy dog.
The quick brown fox jumped over the lazy dog.

\chapter{The Fox and Dog}
The quick brown fox jumped over the lazy dog.
The quick brown fox jumped over the lazy dog.
The quick brown fox jumped over the lazy dog.
The quick brown fox jumped over the lazy dog.
The quick brown fox jumped over the lazy dog.
The quick brown fox jumped over the lazy dog.
The quick brown fox jumped over the lazy dog.
The quick brown fox jumped over the lazy dog.

\section{The Fox Jumps Over}
The quick brown fox jumped over the lazy dog.
The quick brown fox jumped over the lazy dog.
The quick brown fox jumped over the lazy dog.
The quick brown fox jumped over the lazy dog.
The quick brown fox jumped over Uroborus (Figure \ref{uroborus}).
The quick brown fox jumped over the lazy dog.

The quick brown fox jumped over the lazy dog.
The quick brown fox jumped over the lazy dog.
The quick brown fox jumped over the lazy dog.
The quick brown fox jumped over the lazy dog.
The quick brown fox jumped over the lazy dog.
The quick brown fox jumped over the lazy dog.
The quick brown fox jumped over the lazy dog.
The quick brown fox jumped over the lazy dog.
The quick brown fox jumped over the lazy dog.
The quick brown fox jumped over the lazy dog.
The quick brown fox jumped over the lazy dog.

%\vspace{-7mm}
\begin{figure}
\centering
\includegraphics[height=9.2cm,width=13.2cm]{uroboros.pdf}
\vspace{-30mm}
\caption{An alternative hierarchy of the algorithms.}
\label{uroborus}
\end{figure}

The quick brown fox jumped over the lazy dog.
The quick brown fox jumped over the lazy dog.
The quick brown fox jumped over the lazy dog.
The quick brown fox jumped over \cite{ckt} the lazy dog.
The quick brown fox jumped over the lazy dog.
The quick brown fox jumped over the lazy dog.
The quick brown fox jumped over the lazy dog.
The quick brown fox jumped over the lazy dog.

\section{The Lazy Dog}
The quick brown fox jumped over the lazy dog.
The quick brown fox jumped over the lazy dog.
The quick brown fox jumped over the lazy dog.

The quick brown fox jumped over the lazy dog.
The quick brown fox \cite{am97} jumped over the lazy dog.
The quick brown fox jumped over the lazy dog.
The quick brown fox jumped over the lazy dog.
The quick brown fox jumped over the lazy dog.
The quick brown fox jumped over the lazy dog.

%%%%%%%%%%%%%%%%
%              %
%  APPENDICES  %
%              %
%%%%%%%%%%%%%%%%
\begin{appendices}

\chapter{Running the Programs}
An example of running from the command line is as follows:
\begin{verbatim}
      > java MaxClique BBMC1 brock200_1.clq 14400
\end{verbatim}
This will apply $BBMC$ with $style = 1$ to the first brock200 DIMACS instance allowing 14400 seconds of cpu time.

\chapter{Generating Random Graphs}
\label{sec:randomGraph}
We generate Erd\'{o}s-R\"{e}nyi random graphs $G(n,p)$ where $n$ is the number of vertices and
each edge is included in the graph with probability $p$ independent from every other edge. It produces
a random graph in DIMACS format with vertices numbered 1 to $n$ inclusive. It can be run from the command line as follows to produce 
a clq file
\begin{verbatim}
      > java RandomGraph 100 0.9 > 100-90-00.clq
\end{verbatim}
\end{appendices}

%%%%%%%%%%%%%%%%%%%%
%   BIBLIOGRAPHY   %
%%%%%%%%%%%%%%%%%%%%

\bibliographystyle{plain}
\bibliography{bib}

\end{document}
